\documentclass[12pt]{extarticle}
\usepackage[utf8]{inputenc}
\usepackage{amsmath}
\usepackage{cite}
\usepackage{url}
\usepackage{hyperref}
\usepackage{lipsum}
\usepackage{minted}
\usepackage{csquotes}
\usepackage{indentfirst}
\usepackage{color,soul}
\usepackage[a4paper, top=0.1cm, bottom=0.1cm, left=1cm, right=1cm]{geometry}

\title{Project Proposal: Punched Card Reader}
\author{\href{patrick_mccann@brown.edu}{Patrick McCann}, \href{yiru_liu1@brown.edu}{Yi Lyo}, \href{odomsathshia_tep@brown.edu}{Alexander Thaep}, \href{zhiwen_zheng@brown.edu}{Zhiwen \enquote{Michael} Zheng} \\ (in alphabetical order)}
\date{\today}

\begin{document}

\maketitle
% \vspace{-0.5cm}

\subsection*{High-Level Overview}
Our project proposes the design and implementation of an Arduino-based punched card reader.
It aims to emulate historical slow I/O devices with a pedagogical emphasis for CSCI 1670 (Operating Systems).
The reader will pull a punched card through an array of optical sensors, leveraging Pulse-Width Modulation (PWM) to manage the speed of the card's movement.
The analog readings from the sensors will be interpreted using the Arduino's Analog-to-Digital Converter (ADC) to determine the presence or absence of holes in the card.
An interrupt service routine (ISR), triggered by a sensor that detects whether a card is inserted, will initiate the reading process.
The processed data will be transmitted to a host local machine as a string through serial communication.
A watchdog timer will be implemented as a fail-safe mechanism to reset the system in case of malfunctions and ensure the system's reliability.

\subsection*{Final Deliverables}


\subsection*{Component Descriptions}


\subsection*{Uncertainties}


\subsection*{Preliminary List of Components to be Ordered}


\subsection*{References}


% If we include the "plain" references in the previous subsection,
% we may not (or still may) need to include the bibliography.
\bibliographystyle{plain}
\bibliography{project_proposal}

\end{document}
