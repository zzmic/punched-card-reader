\documentclass[12pt]{extarticle}
\usepackage[utf8]{inputenc}
\usepackage{amsmath}
\usepackage{cite}
\usepackage{url}
\usepackage{hyperref}
\usepackage{lipsum}
\usepackage{minted}
\usepackage{csquotes}
\usepackage{indentfirst}
\usepackage{color,soul}
\usepackage[a4paper, top=0.1cm, bottom=0.1cm, left=1cm, right=1cm]{geometry}
\usepackage{titlesec}
\titlespacing*{\subsection}{0pt}{1ex}{1ex}

\title{Project Proposal: Punched Card Reader}
\author{Yi Lyo, Patrick McCann, Alexander Thaep, Zhiwen \enquote{Michael} Zheng \\ (in alphabetical order)}
\date{\today}

\begin{document}

\maketitle
\vspace{-1cm}

\subsection*{Project Overview}
Our project proposes the design and implementation of an Arduino-based punched card reader.
It aims to emulate historical slow I/O devices with a pedagogical emphasis for CSCI 1670 (Operating Systems).
The reader will pull a punched card through an array of optical sensors, leveraging Pulse-Width Modulation (PWM) to manage the speed of the card's movement.
The analog readings from the sensors will be interpreted using the Arduino's Analog-to-Digital Converter (ADC) to determine the presence or absence of holes in the card.
An interrupt service routine (ISR), triggered by a sensor that detects whether a card is inserted (i.e., the presence of a card), will initiate the reading process.
The processed data will be transmitted to a host local machine as a string through serial communication.
A watchdog timer will be implemented as a fail-safe mechanism to reset the system in the event of malfunctions, thereby ensuring the system's reliability.

\subsection*{Design Rationale}
Our project will leverage the Arduino's built-in features, including \textit{Pulse-Width Modulation (PWM)} to control the speed of the card reader and an \textit{Analog-to-Digital Converter (ADC)} to read the analog signals from the optical sensors.
Based on an event-driven approach, the system will utilize an \textit{Interrupt Service Routine (ISR)} to handle card insertion events, thereby initiating the reading process without requiring continuous polling.
To ensure system reliability, a \textit{watchdog timer} will be implemented as a fail-safe fallback to reset the system in case of malfunctions.
Once the data is processed, it will be transmitted to a host local machine via \textit{serial communication}, making the information accessible and fulfilling the device's I/O functionality.

\subsection*{Deliverables and Fulfillment Criteria}


\subsection*{Hardware and Testing Plan}


\subsection*{Uncertainties}


\subsection*{Preliminary List of Components to be Ordered}


\subsection*{References}


% If we include the "plain" references in the previous subsection,
% we may not (or still may) need to include the bibliography.
\bibliographystyle{plain}
\bibliography{project_proposal}

\end{document}
