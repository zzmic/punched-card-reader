\documentclass[12pt]{extarticle}
\usepackage[utf8]{inputenc}
\usepackage{amsmath}
\usepackage{cite}
\usepackage{url}
\usepackage{hyperref}
\usepackage{lipsum}
\usepackage{minted}
\usepackage{csquotes}
\usepackage{indentfirst}
\usepackage{color,soul}
\usepackage[a4paper, top=0.1cm, bottom=0.1cm, left=1cm, right=1cm]{geometry}
\usepackage{titlesec}
\titlespacing*{\subsection}{0pt}{1ex}{1ex}

\title{Project Proposal: Punched Card Reader}
\author{Yi Lyo, Patrick McCann, Alexander Thaep, Zhiwen \enquote{Michael} Zheng \\ (in alphabetical order)}
\date{\today}

\begin{document}

\maketitle
\vspace{-1cm}

\subsection*{Project Overview}
Our project proposes the design and implementation of an Arduino-based punched card reader.
It aims to emulate historical slow I/O devices with a pedagogical emphasis for CSCI 1670 (Operating Systems).
The reader will pull a punched card through an array of optical sensors, leveraging Pulse-Width Modulation (PWM) to manage the speed of the card's movement.
The analog readings from the sensors will be interpreted using the Arduino's Analog-to-Digital Converter (ADC) to determine the presence or absence of holes in the card.
An interrupt service routine (ISR), triggered by a sensor that detects whether a card is inserted (i.e., the presence of a card), will initiate the reading process.
The processed data will be transmitted to a host machine as a string through serial communication.
A watchdog timer will be implemented as a fail-safe mechanism to reset the system in the event of malfunctions, thereby ensuring the system's reliability.

\subsection*{Design Rationale}
Our project will leverage the Arduino's built-in features, including \textit{Pulse-Width Modulation (PWM)} to control the speed of the card reader and an \textit{Analog-to-Digital Converter (ADC)} to read the analog signals from the optical sensors.
Based on an event-driven approach, the system will utilize an \textit{Interrupt Service Routine (ISR)} to handle card insertion events, thus initiating the reading process without requiring continuous polling.
To ensure system reliability, a \textit{watchdog timer} will be implemented as a fail-safe fallback to reset the system in case of malfunctions. Once the data is processed, it will be transmitted to a host machine via \textit{serial communication}, making the information accessible and fulfilling the device's I/O functionality.

\subsection*{Deliverables and Fulfillment Criteria}
A fully assembled, self-contained (connected to a host machine) punched card reader prototype will be presented. 
The prototype will consist of an Arduino Uno R4 WiFi board, an array of optical sensors, and a low-power motor, together with the wires and other auxiliary components that integrate them into a physical entity. 
The project's success will be evaluated based on explicit criteria. 
When a punched card is inserted into the device, the prototype should automatically detect the card through (one of) the optical sensors, pull the card through the sensor array at a controlled speed through the low-power motor, and transmit the data encoded by the punches to the host machine through serial communication. 
On the host machine end, the string displayed in its serial monitor should be consistent with the pattern of holes on the physical card that was read, which, together with the previous behaviors, demonstrates the device's functionality as a reliable I/O peripheral.

\subsection*{Hardware and Testing Plan}
\hl{TODO(zzmic): Explain, \textbf{in one to two sentences}, how we will know that each of the components is working.} \\
The following specifies the hardware components, other than the wires that connect them, that we plan to use and the testing plans we expect to implement:
\begin{itemize}
    \item \textit{Arduino Uno R4 WiFi boards:}
    \item \textit{Array of optical sensors:}
    \item \textit{Low-power motor:}
\end{itemize}

\subsection*{Uncertainties}
There are several challenges that we anticipated, especially in electromechanical integration. 
On the hardware side, achieving precise but necessary physical alignment between the optical sensor array and the punched card rows and/or columns, while ensuring that the low-power motor pushes the card forward at a reasonable speed without deviating the card away from the track, seems to be a hurdle. 
On the software side, synchronizing the reading from the ADC and the card's physical position may require non-trivial timing analysis. 
A fallback solution is to discard the motor and instead manually pull the card to the correct position before reading the immediately following punch, henceforth alleviating both hardware and software concerns.

\subsection*{Preliminary List of Components to be Ordered}
We have a few components, in addition to the Arduino boards and the electronics kits, in mind that need to be ordered:
\begin{itemize}
    \item \textit{An array of optical sensors} that (1) detect and read the punches/holes of a punched card and (2) transmit the sensing information to an Arduino board that the sensors are connected to.
    \item \textit{A low-power motor} that drives a punched card forward in a reasonable speed.
\end{itemize}

\subsection*{References}
\hl{TODO(zzmic): Add references to the terminologies and components that we're going to use, especially for the \textbf{Hardware and Testing Plan} and \textbf{Preliminary List of Components to be Ordered} subsections.} \\

% If we include the "plain" references in the previous subsection,
% we may not (or still may) need to include the bibliography.
% \bibliographystyle{plain}
% \bibliography{project_proposal}

\end{document}
